\section{Methodology}
\label{sec:methodology}

Although there is no universally accepted definition of what it means for a
Haskell rewrite rule to be correct; intuitively any notion of correctness
requires that application of a rule should not change the meaning of a program.
Therefore, we focus on this aspect of correctness: we define a rule to be
\textit{correct} iff applying the rule preserves the semantics of the program.

In this section, we describe the methodology that \Rulecheck uses to test the
correctness of rewrite rules.

\subsection{Property-Based Testing of Rewrite Rules}\label{sec:methodology:testgen}

Intuitively, one could test the correctness of rewrite rule applications for a
program by compiling two versions: one with rewrite rules enabled, and one with
rules disabled. If the output of the two versions differ, then one of the
applied rules must be incorrect.

Unfortunately, generating a test program to trigger a specific rewrite rule is
not straightforward, because including a term in the program that matches a
specific rewrite rule does not ensure that GHC will apply the rule to that term.
For example, the compiler could first perform other syntactic transformations on
the term (such as inlining), that would change the structure of the term so that
the rule no longer applies. Or, it's possible that another rewrite rule (for
example, one defined in the standard library) could first apply to a subterm,
hence preventing the intended rule application.

Therefore, for \Rulecheck we adopted an alternative approach. Instead of relying
on GHC to apply a rewrite rule, we instead bypass the rewrite mechanism
entirely, by constructing a property-based test \textit{derived from} the
rewrite rule. While this approach does not directly test rewrite rule
application (and would therefore fail to identify issues due to bugs in GHC's
rewrite engine, for example); it enables direct testing of the transformation
defined by the rule.

\Rulecheck constructs the property-based test, directly from the syntax of the
rule itself. Conceptually, the LHS and RHS of a rule each define a function from
the free variables of the rule, to their corresponding expression. Concretely,
the given a rewrite rule \texttt{forall \rbinders~.~\rlhs = \rrhs}, we can
construct a pair of functions \texttt{flhs} and \texttt{frhs} as follows:

\begin{minted}{haskell}
flhs x1, ..., xn = lhs
frhs x1, ..., xn = rhs
\end{minted}

The property-based test for the rule, therefore, should check that \texttt{flhs}
and \texttt{frhs} yield semantically equivalent values when applied to the same
inputs. Testing this property requires generating suitable input values (i.e.
values for $\rbinders$), and method for determining semantic equivalence of the
result values at runtime. In the following sections, we describe how \Rulecheck
fulfills these requirements.

\subsection{Generating Test Inputs}\label{sec:methodology:testinput}

\begin{figure}
  \begin{minted}{haskell}
goal :: Either Char Int -> String
goal = \case
  Left y  -> showChar y
  Right z -> showInt z
  \end{minted}
  \caption{An example term that could be used for testing a rule with a free
variable of type \texttt{Either Char Int -> String}. In \Rulecheck, we
synthesize such terms from their type definition.}
  \label{fig:synthexample}
\end{figure}

Haskell does not impose any requirements on the binders \rbinders in a rewrite
rule. Therefore, to test an arbitrary rule, we need a means to generate a term
of potentially arbitrary type.

Generating terms for primitive types (i.e. \texttt{Int}, \texttt{Bool}, etc) is
relatively straightforward. However, as Haskell is a functional language, it is
also necessary to construct arbitrary terms of function type. For example,
consider a rule with a free variable of type \texttt{Either Char Int -> String},
i.e. a function that takes as input either a \texttt{Char} or an \texttt{Int}
value and returns a \texttt{String}. Generating a term of
that type is analagous to synthesize a program from the type specification; for
this reason, we approach the challenge of generating test values using a
\textit{program synthesis} approach.


\subsection{Comparing Test Outputs}\label{sec:methodology:testoutput}

Because Haskell rewrite rules can apply to terms of any type; our methodology
requires that we have some way to determine (or at least, approximate), whether
two values of some arbitrary type are semantically equivalent.

Unfortunately, there does not appear to be a universally applicable way to
determine semantic equivalence at runtime. Haskell provides a typeclass
\texttt{Eq} for user-defined equality, however, this is not implemented by all
types. Furthermore, it's not clear that the implementation will necessarily
correspond to semantic equality. For example, the \texttt{Eq} implementation for
floating point numbers defines \NaN to not equal any other value, including itself.

Alternatively, one could imagine using some intrinsic notion of equality, for
example, a value's representation in memory, to determine semantic equality.
Unfortunately, this is also not suitable in all cases. For example, as Haskell
is a functional language, it is possible to define a rewrite rule to rewrite a
function-typed expression with a more efficient version (typically, a
specialized version of a more general function). Clearly, the original and
replaced function will have different internal representations, even if they
always produce equivalent results.

\ZGTODO{Explain our solution, handling of functions, NaN special case, using synthesis}

Generating tests for rewrite rules comes with its own complications. First, rewritten expressions are intended to be \textit{semantically equivalent} to their source expressions, and what that means varies from library to library. For some, this is true contextual equivalence, and for others this is numerical approximation. To make matters worse, a rewrite may not even preserve user-defined boolean equality (from \texttt{class Eq}) - if that is even defined for the type! So in short, test generators need to remain flexible in the degree and manner in which they assert equality.

The flexibility of the rewrite system is another challenge. One of the few syntactic restrictions of rewrite rules is that ``the left hand side of a rule must consist of a top-level variable applied to arbitrary expressions'' \cite{userguide} - in so many words, rewrite rules can match arbitrary function applications, and transform terms of arbitrary types. These types are not required to implement any particular interfaces, nor are they even required to represent data! They can be function-typed and are eta-expanded as needed. Therefore, test generators need to be prepared to generate terms of a variety of types to act as inputs to or consume outputs from rule applications.

Finally, though parametricity greatly broadens the applicability of rules, it presents difficulties in test generation. Test generators must choose monotypes to instantiate rule polytypes, taking care that our choices implement any required typeclass constraints.

In summary, any test generator for rewrite rules must implement the following features:

\begin{itemize}
  \item Given a type, generate an arbitrary values of (instantiations of) that type
  \item Given two arbitrary values of the same type, compare them for semantic equality
\end{itemize}
