\section{Evaluation}
\label{sec:evaluation}
{

We evaluated \Rulecheck with respect to the following research question:
\textit{Is Rulecheck able to find instances of rewrite rules, defined in
real-world Haskell libraries, that are capable of changing program behaviour?}.

To answer this question, we tested \Rulecheck on packages from
\Hackage, the Haskell package repository. Our first step was to identify the
packages on \Hackage that contained rewrite rules, and identify a subset of the
packages that we considered to be testable.

\newcommand{\totalpackages}{2,912\xspace}  % jq length packages.json
\newcommand{\rulespackages}{123\xspace}    % jq length listings.json
\newcommand{\candidatepackages}{94\xspace} % ls package-tests | wc -l

We began by downloading all packages from Hackage, using the Stackage package
repository resolver \texttt{lts-19.27}; this yielded \totalpackages packages.
Then, we identified the packages containing rewrite rules by using a simple
regular expression to determine which packages contain at least on source files
with a \texttt{\{-\# RULES \#-\}} pragma (\rulespackages packages). We then
attempted to compile each package with GHC version \rcghcver, and excluded
packages that did not compile\footnote{Compilation errors indicated that either
the package was broken, or, more commonly, that the package required external
dependencies that were not available on our test systems.}. After excluding
packages that we deemed untestable, we were left with \candidatepackages
candidate packages for consideration. While this number represents only a small
package of the total packages on \Hackage, these packages are often used as
dependencies for the other packages; therefore, correctness issues in these
packages could also impact the correctness on the packages that rely on
them.\ZGTODO{If time, figure out how many packages on Hackage have one of these
as a transitive dependency.}

\begin{figure}
  \begin{tabular}{ |l|c|c| }
    \hline
    \makecell{\textbf{Package Name}}  & \makecell{\textbf{\# Tests} \\ \textbf{Generated}} & \makecell{\textbf{\# Test} \\ \textbf{Failures}}\\
    \hline
    \texttt{arithmoi} & 11 & 3\\
    \texttt{basement} & 56 & 6\\
    \texttt{bits} & 3 & 2\\
    \texttt{bytestring-strict-builder} & 1 & 0\\
    \texttt{fast-math} & 72 & 42\\
    \texttt{ListLike} & 1 & 0\\
    \hline
  \end{tabular}
  \caption{Results of our evaluation on six Haskell packages. One test is
generated for each rewrite rule in the package. A test failure occurs when a
counterexample is found for the property-based test derived from a rewrite rule.}
  \label{fig:evalresults}
\end{figure}

At the time of writing, we have used \Rulecheck on six candidate packages; four
of the generated test suites had at least one failing tests indicating that a
rewrite rule in the package may be able to change program behaviour. The results
are shown in \figref{evalresults}. In the remainder of this section, we will
focus on the testing results of one such package, \fastmath.

\subsection{Case Study: \fastmath}

The \fastmath\footnote{https://hackage.haskell.org/package/fast-math} package
contains rewrite rules that implement compile-time optimizations for
floating-point calculations. Although most of the included optimizations are
sound in a mathematical sense (i.e. if they were applied to real numbers instead
of floating-point numbers), the library does not attempt to strictly adhere to
Haskell's floating-point semantics. Therefore, this library is especially
interesting for \Rulecheck, as we should expect some of the rules to change
program behaviour.

Indeed, \Rulecheck was able to identify many instances where such changes were
possible. Of the 72 rules defined in \fastmath, \Rulecheck was able to find
failing test cases for 42 of them. We then performed a manual analysis of the
failing test cases and found that they fell into two categories: calculations
involving NaN values, and small numerical discrepencies.

\subsubsection{Calculations involving \NaN values}

\begin{figure}
  \begin{minted}[linenos]{text}
divideFloatz20U0z20Ux_6_1: FAIL
*** Failed! Falsified (after 1 test):
0.0
LHS was NaN, RHS was 0.0
  \end{minted}
  \caption{Counterexample found by \Rulecheck for the rule \texttt{forall x. 0.0
/ x = 0}. The \texttt{0.0} on line 3 indicates that the LHS and RHS
of the rule evaluate to different values when substituting \texttt{0} for \texttt{x}.}
  \label{fig:div0}
\end{figure}

Several of the test cases failed when considering rule applications for
expressions that would normally produce a \NaN value. One such failure is
presented in \figref{div0}, for the test case generated from the rule
\texttt{forall x. 0.0 / x = 0}. In the case when \texttt{x} itself is
\texttt{0}, applying the rule would rewrite the expression \texttt{0.0 / 0} to
\texttt{0}. However, in Haskell, the expression \texttt{0.0 / 0} evaluates to
\NaN; not \texttt{0}. Therefore, applying this rule would could change the
behaviour of the program. \Rulecheck was able to identify that applying this
rule would yield a different result when \texttt{x} was \texttt{0}, resulting in
a failing test case.

\subsubsection{Numerical Discrepencies}

\begin{figure}
  \begin{minted}[linenos]{text}
floatz20Uztz2cUzpz20Udistributez20UA_37_1:  FAIL
*** Failed! Falsified (after 3 tests and 14 shrinks):
(0.1,-1.0,0.1)
LHS was -0.09, RHS was -0.089999996
  \end{minted}
  \caption{Counterexample found by \Rulecheck for the rule \texttt{forall x y1
y2. (x * y1) + (x * y2) = x * (y1 + y2)}. The output on line 3 indicates
that the LHS and RHS of the rule evaluate to different values when substituting
\texttt{0.1}, \texttt{-1.0}, and \texttt{0.1} for \texttt{x}, \texttt{y1}, and \texttt{y2} respectively.}
  \label{fig:dist}
\end{figure}

It is well-known that operations on floating-point numbers may not yield precise
results due to imprecision in the representation. Therefore, arithmetic
expressions that would be equivalent in the mathematical domain may nonetheless
produce different results when applied to floating-point numbers. This phenomena
is observed in several cases in the \texttt{fast-math}.

For example, the rule \texttt{forall x y1 y2. (x * y1) + (x * y2) = x * (y1 +
y2)} defines an optimization based on a distributivity property, changing a
computation involving two multiplications to one with only a single
multiplication. However, as shown in \figref{dist}, applying this rule can lead
to small numerical differences in the resulting computation.

\smallskip
As for the remaining 30 rules, \Rulecheck cannot definitively prove that they
are correct. However, because it is unable to find a counterexample when
considering many different applications of each rule, this result nonetheless
serves to increase the confidence in the correctness of those rules.
