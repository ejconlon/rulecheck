\section{Related work}
\label{sec:related}

There is a long history of using types to guide automated testing of functional programs. \Quickcheck \cite{claessen2000quickcheck}, which we use extensively in this project, allows users to construct test input generators to validate lightweight logical specifications. These generators are commonly derived using typeclasses, meaning that the structure of a generator has a strong correspondence to structure of the type. \Quickcheck supports input shrinking, but this feature is not always useful in derived implementations. Projects such as \texttt{SmallCheck} \cite{runciman2008smallcheck} try to fix this by changing their strategy from generation to enumeration. Others like \texttt{Hedgehog} \cite{hedgehog} integrate shrinking directly into their generators, ensuring that shrinking always works, at the cost of complicating generator definitions.

Types have also been used extensively to direct program synthesis. In this work, we have no requirements for our generated programs aside from type compatibility, but many works describe the type- \textit{and} example-directed synthesis. \cite{osera2015type, feser2015synthesizing} Types constrain the search space to valid programs, and examples constrain the search space to relevant programs. Where fully automated search is insufficient, offering the user a selection of tactics can guide the system toward an answer. \cite{delahaye2000tactic} Automata-based representations \cite{koppel2022searching} are another alternative to tactics-based search for enumeration of program spaces with constraints.

In addition to their use in defining program transformations, rewrite rules
appear more broadly in the context of term rewriting; an overview on the subject
is presented in \cite{klop1990term}. Early applications of term rewriting
focused on its use to solve equational reasoning problems
\cite{knuth1983simple}. Rewriting also appears as a tactic in interactive
theorem provers, such as Coq\cite{coq} and Lean\cite{lean}, where it is used to
automate axiom instantiation and therefore simplify proofs. Term rewriting
systems have also been used to interpret and prove properties about programs
\cite{Dershowitz1985ComputingWR}.

There is a wide range of prior work on ensuring the correctness of rewrite-rule
based optimizations. Notably, program generation has uncovered several bugs in
the internal rewriting rules of SMT solvers. \cite{winterer2020validating}.
Proving the correctness of rewriting-based transformations has previously been
applied to Cobolt, a DSL for defining compile-time optimizations
\cite{lerner2003automatically}. Some program transformatiosn implemented as
Haskell rewrite rules, such as short cut fusion\cite{shortcutfusion}, have been
proven to preserve program semantics\cite{johann2003short}. However, to the best
of our knowledge, there is no previous work on testing the correctness of
Haskell rewrite rules directly.

In \Rulecheck, we focused on the semantics-preserving aspect of rewrite rules.
However, confluence and termination are often considered as desirable properties
of rewrite systems: in particular, the combination of the two is sufficient to
ensure that every expression can be rewritten to a unique normal form. These
aspects of correctness are essentially orthogonal: semantic correctness is a
property of individual rewrite rules; while confluence and termination are
properties about a set of rewrite rules. Therefore, proving confluence and
termination is out of scope for our technique. The utility
\texttt{GSOL}\cite{hamana2022gsol} can be used to prove confluence of rewrite
rules in Haskell.
