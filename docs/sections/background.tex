\section{Background}
\label{sec:background}

In this section, we provide a high-level conceptual overview of rewrite rules in
Haskell. In particular, we omit details of rewriting that are not relevant for
\Rulecheck (such as phase-control) in the interest of clarity. Readers
interested in the full syntax and semantics of rewrite rules in Haskell are
suggested to consult the GHC documentation \cite{userguide}.

A rewrite rule is declared using an annotation of the form:\\ \rrule,\\ where \rbinders is a (possibly-empty) list
of variable names, and \rlhs and \rrhs (the left and
right-hand sides of the rule, respectively) are Haskell expressions. The
expressions on each side of the rule can refer to functions in scope as well as
the variable names in \rbinders; the referenced variables are called
the \textit{free variables} of that side.

At compile time, GHC analyzes expressions in the abstract syntax tree, and
replaces expressions that \textit{match} the left-hand side of a rewrite rule
with their rewritten equivalents. An expression $\texttt{e}$ matches the
left-hand side \rlhs iff \rlhs can be made syntactically equal to $\texttt{e}$
by substituting the free variables in \rlhs with appropriate expressions. If
there is a match, the compiler then applies the substitution to the
right-hand-side of the rule, and replaces \texttt{e} with the resulting
expression.

\ZGTODO{Consider mentioning that rules are applied multiple times, order of
  consideration not defined, considerations due to inlining, possible disabiling
due to optimization flags, termination, confluence, etc.}

\ECTODO{Also mention that information in rewrite rules is lost after their application - we have to look at program source}
